%% ============================================================
%% PASTE THE FOLLOWING INTO YOUR PREAMBLE (once per document)
%% Engine required: XeLaTeX or LuaLaTeX
%% ============================================================
%%
%% % ---- Font -------------------------------------------------
%% % XeLaTeX / LuaLaTeX — Overleaf: Menu > Compiler > XeLaTeX
%% \usepackage{fontspec}
%% \setmonofont[Scale=0.85]{JetBrains Mono}
%%
%% % ---- Colors -----------------------------------------------
%% \usepackage{xcolor}
%% \definecolor{vscBackground}{RGB}{0,29,74}
%% \definecolor{vscFrameOuter}{RGB}{21,21,22}
%% \definecolor{vscGradStart}{RGB}{28,177,242}
%% \definecolor{vscGradEnd}{RGB}{0,89,255}
%% \definecolor{vscDefault}{RGB}{154,214,254}
%% \definecolor{vscKeyword}{RGB}{82,157,218}
%% \definecolor{vscFunction}{RGB}{220,220,168}
%% \definecolor{vscNumber}{RGB}{180,205,167}
%% \definecolor{vscString}{RGB}{206,145,120}
%% \definecolor{vscComment}{RGB}{141,161,185}
%% \definecolor{vscOperator}{RGB}{250,250,250}
%% \definecolor{vscBracket}{RGB}{219,215,0}
%% \definecolor{vscType}{RGB}{78,201,176}
%% \definecolor{vscDecorator}{RGB}{220,220,168}
%% \definecolor{vscLineno}{RGB}{124,128,131}
%% \definecolor{vscSelection}{RGB}{38,79,120}
%% \definecolor{vscBtnRed}{RGB}{255,95,87}
%% \definecolor{vscBtnYellow}{RGB}{254,188,46}
%% \definecolor{vscBtnGreen}{RGB}{40,200,64}
%%
%% % ---- Packages (EXACT order matters) ----------------------------
%% \usepackage{tikz}
%% \usetikzlibrary{calc,shadings}
%% %% IMPORTANT: load tcolorbox with [most], then IMMEDIATELY call
%% %% \tcbuselibrary. Do NOT split these two lines apart.
%% \usepackage[most]{tcolorbox}
%% \tcbuselibrary{skins,breakable}
%%
%%
%% NOTE: No \tcbset{codewindow} needed — all tcolorbox options are
%% embedded directly in each \begin{tcolorbox}[...] block.
%% Remove any old codewindow/codepanel \tcbset entries from preamble.
%% ============================================================
%% Auto-generated by export_tikz.py — source: counter_test.py
%%
%% Single flat breakable tcolorbox.
%% The blue gradient is painted as a TikZ overlay (behind content) on every
%% page segment; the dark panel fill sits on top via colback=vscBackground.
%% This avoids the nested-box page-break problem entirely.
\begin{tcolorbox}[
  enhanced, breakable,
  arc=12pt, outer arc=12pt,
  boxrule=0pt, frame hidden,
  toprule at break=0pt, bottomrule at break=0pt,
  colback=vscBackground,
  colupper=vscDefault,
  left=18pt, right=18pt, top=18pt, bottom=18pt,
  % Gradient underlay — painted behind content on every page segment
  underlay={%
    \fill[left color=vscGradStart, right color=vscGradEnd, shading angle=135,
           rounded corners=12pt]
      (frame.south west) rectangle (frame.north east);
    \fill[vscBackground, rounded corners=8pt]
      ([xshift=18pt,yshift=15pt]frame.south west)
      rectangle
      ([xshift=-18pt,yshift=-15pt]frame.north east);
  },
]

  % Tight line spacing — zero paragraph skip, compact baseline
  \parskip=0pt\relax
  \baselineskip=1.35ex\relax

  % macOS chrome bar: traffic lights + filename (first page only)
  {\tiny\ttfamily%
  \tikz[baseline=-0.6ex]{%
    \fill[vscBtnRed]    (0,0)     circle (4pt);
    \fill[vscBtnYellow] (12pt,0)  circle (4pt);
    \fill[vscBtnGreen]  (24pt,0)  circle (4pt);
  }\hspace{10pt}%
  \textcolor{vscDefault}{counter\_test.py}}\par
  \vspace{3pt}%
  {\color{vscLineno}\hrule height 0.3pt}%
  \vspace{2pt}%

  % Code lines — each is an independent paragraph so tcolorbox can break here
  \noindent\makebox[2.0em][r]{{\tiny\ttfamily\textcolor{vscLineno}{1}}}\hspace{4pt}{\tiny\ttfamily \textcolor{vscDefault}{function}\textcolor{vscDefault}{\kern\fontdimen2\font}\textcolor{vscDefault}{Counter}\textcolor{vscBracket}{(}\textcolor{vscBracket}{)}\textcolor{vscDefault}{\kern\fontdimen2\font}\textcolor{vscBracket}{\{}}\par
  \noindent\makebox[2.0em][r]{{\tiny\ttfamily\textcolor{vscLineno}{2}}}\hspace{4pt}{\tiny\ttfamily \textcolor{vscDefault}{\kern\fontdimen2\font\kern\fontdimen2\font}\textcolor{vscDefault}{const}\textcolor{vscDefault}{\kern\fontdimen2\font}\textcolor{vscBracket}{[}\textcolor{vscDefault}{count}\textcolor{vscOperator}{,}\textcolor{vscDefault}{\kern\fontdimen2\font}\textcolor{vscDefault}{setCount}\textcolor{vscBracket}{]}\textcolor{vscDefault}{\kern\fontdimen2\font}\textcolor{vscOperator}{=}\textcolor{vscDefault}{\kern\fontdimen2\font}\textcolor{vscDefault}{createSignal}\textcolor{vscBracket}{(}\textcolor{vscNumber}{0}\textcolor{vscBracket}{)}\textcolor{vscOperator}{;}}\par
  \noindent\makebox[2.0em][r]{{\tiny\ttfamily\textcolor{vscLineno}{3}}}\hspace{4pt}{\tiny\ttfamily\strut}\par
  \noindent\makebox[2.0em][r]{{\tiny\ttfamily\textcolor{vscLineno}{4}}}\hspace{4pt}{\tiny\ttfamily \textcolor{vscDefault}{\kern\fontdimen2\font\kern\fontdimen2\font}\textcolor{vscDefault}{setInterval}\textcolor{vscBracket}{(}}\par
  \noindent\makebox[2.0em][r]{{\tiny\ttfamily\textcolor{vscLineno}{5}}}\hspace{4pt}{\tiny\ttfamily \textcolor{vscDefault}{\kern\fontdimen2\font\kern\fontdimen2\font\kern\fontdimen2\font\kern\fontdimen2\font}\textcolor{vscBracket}{(}\textcolor{vscBracket}{)}\textcolor{vscDefault}{\kern\fontdimen2\font}\textcolor{vscOperator}{=}\textcolor{vscOperator}{\textgreater{}}\textcolor{vscDefault}{\kern\fontdimen2\font}\textcolor{vscDefault}{setCount}\textcolor{vscBracket}{(}\textcolor{vscDefault}{count}\textcolor{vscBracket}{(}\textcolor{vscBracket}{)}\textcolor{vscDefault}{\kern\fontdimen2\font}\textcolor{vscOperator}{+}\textcolor{vscDefault}{\kern\fontdimen2\font}\textcolor{vscNumber}{1}\textcolor{vscBracket}{)}\textcolor{vscOperator}{,}}\par
  \noindent\makebox[2.0em][r]{{\tiny\ttfamily\textcolor{vscLineno}{6}}}\hspace{4pt}{\tiny\ttfamily \textcolor{vscDefault}{\kern\fontdimen2\font\kern\fontdimen2\font\kern\fontdimen2\font\kern\fontdimen2\font}\textcolor{vscNumber}{1000}}\par
  \noindent\makebox[2.0em][r]{{\tiny\ttfamily\textcolor{vscLineno}{7}}}\hspace{4pt}{\tiny\ttfamily \textcolor{vscDefault}{\kern\fontdimen2\font\kern\fontdimen2\font}\textcolor{vscBracket}{)}\textcolor{vscOperator}{;}}\par
  \noindent\makebox[2.0em][r]{{\tiny\ttfamily\textcolor{vscLineno}{8}}}\hspace{4pt}{\tiny\ttfamily\strut}\par
  \noindent\makebox[2.0em][r]{{\tiny\ttfamily\textcolor{vscLineno}{9}}}\hspace{4pt}{\tiny\ttfamily \textcolor{vscDefault}{\kern\fontdimen2\font\kern\fontdimen2\font}\textcolor{vscKeyword}{return}\textcolor{vscDefault}{\kern\fontdimen2\font}\textcolor{vscOperator}{\textless{}}\textcolor{vscDefault}{div}\textcolor{vscOperator}{\textgreater{}}\textcolor{vscDefault}{The}\textcolor{vscDefault}{\kern\fontdimen2\font}\textcolor{vscDefault}{count}\textcolor{vscDefault}{\kern\fontdimen2\font}\textcolor{vscKeyword}{is}\textcolor{vscDefault}{\kern\fontdimen2\font}\textcolor{vscBracket}{\{}\textcolor{vscDefault}{count}\textcolor{vscBracket}{(}\textcolor{vscBracket}{)}\textcolor{vscBracket}{\}}\textcolor{vscOperator}{\textless{}}\textcolor{vscOperator}{/}\textcolor{vscDefault}{div}\textcolor{vscOperator}{\textgreater{}}}\par
  \noindent\makebox[2.0em][r]{{\tiny\ttfamily\textcolor{vscLineno}{10}}}\hspace{4pt}{\tiny\ttfamily \textcolor{vscBracket}{\}}}\par

\end{tcolorbox}
\label{code:counter-test}
\begin{center}
\textbf{Code 1:} Counter Test
\end{center}
